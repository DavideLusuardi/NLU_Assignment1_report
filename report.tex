\documentclass[conference]{IEEEtran}
\IEEEoverridecommandlockouts
% The preceding line is only needed to identify funding in the first footnote. If that is unneeded, please comment it out.
\usepackage{cite}
\usepackage{url}
\usepackage[english]{babel}
\usepackage{amsmath,amssymb,amsfonts}
\usepackage{algorithmic}
\usepackage{graphicx}
\graphicspath{ {./images/} }
\usepackage{textcomp}
\usepackage{xcolor}
\def\BibTeX{{\rm B\kern-.05em{\sc i\kern-.025em b}\kern-.08em
    T\kern-.1667em\lower.7ex\hbox{E}\kern-.125emX}}
\begin{document}

\title{NLU Assignment 1 - Report\\
% {\footnotesize \textsuperscript{*}Note: Sub-titles are not captured in Xplore and
% should not be used}
% \thanks{Identify applicable funding agency here. If none, delete this.}
}

\author{\IEEEauthorblockN{Davide Lusuardi}
\IEEEauthorblockA{\textit{Department of information engineering and computer science} \\
\textit{University of Trento}\\
Trento, Italy \\
davide.lusuardi@studenti.unitn.it}
}

\maketitle


\section{Description}
In this report we discuss some implementation choices for the first assignment of the course Natural Language Understanding.

The following functions has been implemented:
\begin{itemize}
    \item \texttt{extract\_path(sentence)}. This function extracts for each token a path of dependency relations from the 
    ROOT to the token. In particular, the function parses the sentence to get a Doc object of spaCy and for each token 
    finds the path of dependency relations from the ROOT to the token.

    The function accepts a sentence of type \texttt{string} and returns a dictionary containing for each token (keys in the dictionary)
    the lists of dependency relations.
\end{itemize}

% The python script has been tested on a linux environment:

\end{document}
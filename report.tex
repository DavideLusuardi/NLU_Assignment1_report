% \documentclass[conference]{IEEEtran}
\documentclass{article}
% \IEEEoverridecommandlockouts
% The preceding line is only needed to identify funding in the first footnote. If that is unneeded, please comment it out.
\usepackage{cite}
\usepackage{url}
\usepackage[english]{babel}
\usepackage{amsmath,amssymb,amsfonts}
\usepackage{algorithmic}
\usepackage{graphicx}
\graphicspath{ {./images/} }
\usepackage{textcomp}
\usepackage{xcolor}
\def\BibTeX{{\rm B\kern-.05em{\sc i\kern-.025em b}\kern-.08em
    T\kern-.1667em\lower.7ex\hbox{E}\kern-.125emX}}
\begin{document}

\title{Report - NLU Assignment 1\\
% {\footnotesize \textsuperscript{*}Note: Sub-titles are not captured in Xplore and
% should not be used}
% \thanks{Identify applicable funding agency here. If none, delete this.}
}

% \author{\IEEEauthorblockN{Davide Lusuardi}
% \IEEEauthorblockA{\textit{Department of information engineering and computer science} \\
% \textit{University of Trento}\\
% Trento, Italy \\
% davide.lusuardi@studenti.unitn.it}
% }

\author{Davide Lusuardi}
% \date{February 2014}

\begin{titlepage}
    \maketitle
\end{titlepage}

% \maketitle


\section{Description}
In this report we present some implementation choices of the first assignment of the course Natural Language Understanding.

The following functions has been implemented.
\begin{itemize}
    \item \texttt{extract\_path(sentence)}. This function extracts for each token a path of dependency relations from the 
    ROOT to the token. We first parse the sentence to get a Doc object of spaCy and then, for each token in the sentence,
    we find the path of dependency relations from the ROOT to the token.

    The function accepts a sentence of type \texttt{string} and returns a dictionary containing for each token (keys in the dictionary)
    the list of dependency relations.

    \item \texttt{extract\_subtree(sentence)}. This function extracts the subtree of each token in the sentence. We first parse 
    the sentence to get a Doc object of spaCy and then, for each token, we extract the subtree using the attribute \texttt{Token.subtree}.

    The function accepts a sentence of type \texttt{string} and returns a dictionary containing for each token (keys in the dictionary)
    a list of tokens, ordered with respect to the sentence, representing the subtree.

    \item \texttt{check\_subtree(sentence, words)}. This function checks if the given list of words forms a subtree of the 
    dependency graph of the sentence.

    The function accepts a sentence of type \texttt{string} and a list of words (list of \texttt{string}). It returns a boolean value, 
    \texttt{True} if the words form a subtree.

    The function extracts all the subtrees of the sentence using the function \texttt{extract\_subtree()} and checks if the list of words
    match a subtree.

    \item \texttt{get\_head(span)}. This function gets the head of the given span. From spaCy documentation, the head token is such that
    \texttt{token.head == token}.

    The function accepts a span of type \texttt{string} and returns a token of type \texttt{spacy.tokens.token.Token}.

    \item \texttt{extract\_nsubj\_dobj\_iobj(sentence)}. 
    
    This function extracts the sentence subject (nsubj), direct object (dobj) and 
    indirect object (iobj) spans from the sentence.

    The function accepts a sentence of type \texttt{string} and returns a dictionary containing \texttt{"nsubj", "dobj", "iobj"} as keys 
    and the relative list of spans as values. If for example there are more subjects in the sentence, the entry "nsubj" will be a list 
    containing all the subject spans. A span is represented as a list of tokens.

    
\end{itemize}

% The python script has been tested on a linux environment:

\end{document}